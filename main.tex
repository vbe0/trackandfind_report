%%%%%%%%%%%%%%%%%%%%%%%%%%%%%%%%%%%%%%%%%
% Academic Title Page
% LaTeX Template
% Version 2.0 (17/7/17)
%
% This template was downloaded from:
% http://www.LaTeXTemplates.com
%
% Original author:
% WikiBooks (LaTeX - Title Creation) with modifications by:
% Vel (vel@latextemplates.com)
%
% License:
% CC BY-NC-SA 3.0 (http://creativecommons.org/licenses/by-nc-sa/3.0/)
% 
% Instructions for using this template:
% This title page is capable of being compiled as is. This is not useful for 
% including it in another document. To do this, you have two options: 
%
% 1) Copy/paste everything between \begin{document} and \end{document} 
% starting at \begin{titlepage} and paste this into another LaTeX file where you 
% want your title page.
% OR
% 2) Remove everything outside the \begin{titlepage} and \end{titlepage}, rename
% this file and move it to the same directory as the LaTeX file you wish to add it to. 
% Then add \input{./<new filename>.tex} to your LaTeX file where you want your
% title page.
%
%%%%%%%%%%%%%%%%%%%%%%%%%%%%%%%%%%%%%%%%%

%----------------------------------------------------------------------------------------
%	PACKAGES AND OTHER DOCUMENT CONFIGURATIONS
%----------------------------------------------------------------------------------------

\documentclass[10pt]{article}

\usepackage[utf8]{inputenc} % Required for inputting international characters
\usepackage[T1]{fontenc} % Output font encoding for international characters
\usepackage{mathpazo} % Palatino font
\usepackage{tocloft,lipsum,pgffor,sectsty}

\setcounter{tocdepth}{3}% Include up to \subsubsection in ToC

% Font changes to ToC content of sectional units
\renewcommand{\cftsecfont}{\LARGE\scshape}
\renewcommand{\cftpartfont}{\normalfont\sffamily\bfseries}% \part font in ToC
\renewcommand{\cftsecfont}{\normalfont\slshape}           % \section font in ToC
\renewcommand{\cftsubsecfont}{\normalfont\itshape}        % \subsection font in ToC
\renewcommand{\cftsubsubsecfont}{\normalfont\small}       % \subsubsection font in ToC

% Font changes to document content of sectional units
\renewcommand{\partfont}{\normalfont\Huge\bfseries}
\renewcommand{\chapterfont}{\normalfont\huge\bfseries}
\renewcommand{\sectionfont}{\normalfont\LARGE\bfseries}


\begin{document}

%----------------------------------------------------------------------------------------
%	TITLE PAGE
%----------------------------------------------------------------------------------------

\begin{titlepage} % Suppresses displaying the page number on the title page and the subsequent page counts as page 1
	\newcommand{\HRule}{\rule{\linewidth}{0.5mm}} % Defines a new command for horizontal lines, change thickness here
	
	\center % Centre everything on the page
	
	%------------------------------------------------
	%	Headings
	%------------------------------------------------
	
	\textsc{\LARGE INF-3910-3-3 - Computer Science Seminar: IoT services with LoRaWAN network and compatible embedded devices and sensors}\\[1.5cm] % Main heading such as the name of your university/college
	
	\textsc{\Large University of Tromsø}\\[0.5cm] % Major heading such as course name
	
	
	%------------------------------------------------
	%	Title
	%------------------------------------------------
	
	
	
	{\huge\bfseries Animal tracker project }\\[0.4cm] % Title of your document

	%------------------------------------------------
	%	Author(s)
	%------------------------------------------------
	
	% If you don't want a supervisor, uncomment the two lines below and comment the code above
	{\Large{Thomas Bye Nilsen\\Valter Berg}}
	%------------------------------------------------
	%	Date
	%------------------------------------------------
	
	\vfill\vfill\vfill % Position the date 3/4 down the remaining page
	
	{\large\today} % Date, change the \today to a set date if you want to be precise
		
	\vfill % Push the date up 1/4 of the remaining page
	
\end{titlepage}

\tableofcontents
\listoffigures
%\listoftables

\clearpage

\section{Introduction}
	This report describes the work that lays the foundation for our project. The project is about tracking certain animals in their natural habitat without hindering their way of living. The motivation for our project is the usefulness of the services provided by the system we develop. 
	\subsection{Problem statement}
		The problem we want to solve is to gather and present data regarding the environment and migration patterns of the animals using the IoT devices. We want to make the IoT devices fit as many species as possible. We will track animals by emitting GPS coordinates, battery voltage and temperature from the wearers. All data will be delivered to a backend system through the LoRaWAN network. We will also make it scalable. 
		\\\\
		We will primarily focus on individually gather the four types of data from an arbitrary number of observational units; we will not focus on creating a mesh network of sensors or the aspect of fault-tolerance. 
		\\\\
		The complexity of the project is not too high for a practical completion. The technology needed is available and there seems to be not too much work to achieve it. In addition, we have been in contact with Rovdata, a company that tracks wild animals for the purpose of population control and positional tracking. We also did contact some farmers wo works with sheep herding to see if they could contribute with any opinions on the design. All the farmers were interested, but they were concerned about a evantual cost of implementing such a system on all their sheeps, as they have 300 - 600 individuals grassing each summer. That is why we also indend to keep the cost as low as possible, which also should be an advantage with using cheap lopy devices an a low cost network such as LoRaWAN. 

	\subsection{Background}
		*words*
		
	\subsection{Member roles}
		The team members are Valter Berg and Thomas Bye Nilsen. The member agree on the tasks that are to be solved. Both members work on the sensors, each of which implements separate features that make up the sensor functionalities. Thomas designs the format for data transmission between the sensors and the backend system and present the data on the backend system. He also designs and makes the boxes that will contain battery and IoT device. Valter works on managing the sensors from the backend system and implements a map functionality. Both members contact potential users in a real-world setting. They also conduct tests on animals. Both members will update each other on the state of the project.
		\\\\
		The rest of the report is laid out as milestone chapters, starting from milestone 2. The milestones are laid out on a time line that incrementally build a solution to the problem. Milestone 2 includes the project design, a high-level priority list of features to implement, technical design and outlining of risks, as well as how they will be mitigated. 
		
		
%		The design part will include an abstract design description, the architecture of the system and the use of the protocols. Each system component will be designed as well, such as the sensors, the communication between the sensors and the backend system and the backend system itself. 
		


\section{Milestone 2}
\subsection{Design decisions}
	As of 19th of February, all (high-level) design decisions are clarified. To gather relevant data, we learn that a Pytrack expansion board and a temperature sensor are suitable for our use because they provide all the features we need. The relevant components are one GPS transmitter, accelerometer sensor and a DS18B20 one-wire digital temperature sensor. It reports degrees in Celsius in the range 55C to 125C (+/-0.5C). The GPS transmitter and the accelerometer sensor are embedded in the Pytrack. We need to focus on the power consumption because animals are in their natural habitat for months at a time, depending on the species. We would like to make the sensors as low-powered as possible because it takes time to replace the battery packages. We also need to focus on the design of the boxes. An important aspect is that they must not bother the wearers. In addition, it must be water-proof and able to withstand nudges and that they don't fall off the wearer. Finally, we need to present the data that are meaningful to the user.
	\\\\
	We prioritize the order in which the features are implemented. The features that must be included are:
	\begin{itemize}  
		\item Gather GPS coordinates, accelerometer data and temperature readings on the senors
		\item Send the gathered data to a MIC Cloud instance provided by Telenor
		\item Send data from MIC Cloud to a back-end system
		\item Implement a back-end system with login sessions 		
		\item Show the GPS coordinates in a map, along with time stamps and temperature readings
		\item Check if accelerometer data has not changed in a given time
	\end{itemize}
	The features that should be included:
	\begin{itemize}  
		\item Two-way communication with MQTT
		\item Show info and previous positions for each device
		\item Notify the user if there is no change in movement for the individuals
		\item Notify if some individuals are in unwanted positions, such as close to village, roads or similar places.
	\end{itemize}
	The features that could be included are:
	\begin{itemize}  
		\item Notify subscribed users on email or phone if interesting events happen
		\item Implement a smart phone application that interfaces with the system
		\item Comprehensive info about each animal.
		\item Show any grassing areas that are beneficial to weight conditions or other factors
	\end{itemize}

\subsection{Overall flow and states}
	So far we have a direction and goal for the project. We have a list for all the sensors we need, an architecture for the system and we also know how to design boxes that will be 3D-printed. In addition, we have setup a development environment and tested the LoPy device by sending some arbitrary data to MIC Cloud. We have also looked into popular development frameworks for implementing a back-end system for presenting the data.

\subsection{Scope out and estimate time for work}
	Taking the current state of the project into consideration, we believe we are on schedule. 










\end{document}



